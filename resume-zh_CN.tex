% !TEX TS-program = xelatex
% !TEX encoding = UTF-8 Unicode
% !Mode:: "TeX:UTF-8"

\documentclass{resume}
\usepackage{zh_CN-Adobefonts_external} % Simplified Chinese Support using external fonts (./fonts/zh_CN-Adobe/)
\usepackage{linespacing_fix} % disable extra space before next section
\usepackage{cite}

\begin{document}
\pagenumbering{gobble} % suppress displaying page number

\name{邵正将}

\basicInfo{
  \email{shaozhengjiang@gmail.com} \textperiodcentered\
  \phone{185-1669-0432} \textperiodcentered\
  \faGithub\ \href{https://github.com/PytLab}{PytLab} \textperiodcentered\
  \faLink\ \href{http://pytlab.org}{pytlab.org}}
 
\section{\faGraduationCap\  教育背景}
\datedsubsection{\textbf{华东理工大学\ 工业催化研究所计算化学中心}, 上海}{2014 -- 至今}
\textit{在读博士研究生}(保送直博)\ 计算化学, 预计 2019 年 6 月毕业
\datedsubsection{\textbf{华东理工大学}, 上海}{2010 -- 2014}
\textit{工学学士}\ 计算机科学与技术(辅修)
\datedsubsection{\textbf{华东理工大学}, 上海}{2010 -- 2014}
\textit{理学学士}\ 材料化学

\section{\faCogs\ 技能}
\begin{itemize}[parsep=0.5ex]
  \item 编程语言: Python, C++, C, ECMAScript, Shell
  \item 操作系统: Linux, macOS, Windows
  \item 开发工具: Vim, Git, CMake, gdb, SWIG
  \item 开发框架: MPI, TensorFlow, sklearn, Flask
  \item 数学基础: 线性代数, 遗传算法, 最优化方法
  \item 测试工具: PyUnit, CppUnit, Googletest
  \item 英语能力: CET6 (568)
\end{itemize}

\section{\faUsers\ 科研项目}

\datedsubsection{\textbf{实现基于遗传算法的反应力场(ReaxFF)的通用优化方法}}{2017年8月 -- 2018年1月}
\role{Python, C++, MPI, Linux}{独自开发, 指导教师: \faLink \href{http://www.ecust.edu.cn/2016/1018/c226a56807/page.htm}{胡培君}教授}
\begin{onehalfspacing}
\begin{itemize}
    \item 实现基于MPI的分布式通用Python遗传算法优化框架\faLink \href{https://github.com/PytLab/gaft}{GAFT}\ (\faStar\ 128 \ \faCodeFork\ 45)
    \item 实现基于MPI的分布式通用C++遗传算法求解器\faLink \href{https://github.com/PytLab/GASol}{GASol}\ (\faStar\ 3\ \faCodeFork\ 1)
  \item 基于GAFT实现反应力场ReaxFF的力场优化程序(未开源, GitHub private repository)
  \item 优化PtO催化体系的反应力场势能面同第一性原理能量均方根误差达到0.2eV以下
\end{itemize}
\end{onehalfspacing}

\datedsubsection{\textbf{基于机器学习算法的催化剂表面物种相互作用的研究}}{2017年9月 -- 至今}
\role{Python, scikit-learn, TensorFlow, jupyter notebook} {指导教师: \faLink \href{http://chem.ecust.edu.cn/2014/1211/c6655a50467/page.htm}{曹宵鸣}副教授}
\begin{onehalfspacing}
\begin{itemize}
    \item 尝试使用不同机器学习模型(SVR, DNN, CNN)构建催化剂二维表面构型与形成能的回归关系
    \item 通过表面二维结构对称性扩大表面构型形成能数据集
    \item 使用自己开发的遗传算法框架GAFT对SVR模型超参数进行优化
    \item 训练的模型对于1-2个物种覆盖的表面形成能预测误差低于0.2eV
    \item 将训练的模型整合进KMC模拟中从而在动力学蒙特卡洛模拟中考虑物种间的相互作用
\end{itemize}
\end{onehalfspacing}

\datedsubsection{\textbf{\emph{ab initio} kMC模拟中克服物种迁移低能垒问题的新方法研究}}{2016年9月 -- 2017年5月}
\role{Python, C++, MPI, Linux} {独自开发, 指导教师: \faLink \href{http://chem.ecust.edu.cn/2014/1211/c6655a50467/page.htm}{曹宵鸣}副教授}
\begin{onehalfspacing}
\begin{itemize}
    \item 实现了Python/C++动力学蒙特卡洛(kMC)模拟库KMCLibX(未开源, GitHub private repository)
    \item 使用MPI接口并行化蒙特卡洛模拟过程进行加速和提升可扩展性
    \item 使用SWIG为kMC C++库添加Python接口提升库的灵活性
    \item 使用time-scale coupling/decoupling 方法克服了物种迁移低能垒问题大幅度加速模拟的时间推进
    \item 完成论文\emph{ab initio kinetic Monte Carlo simulation of CO Oxidation: Methods for overcoming the low-barrier diffusion problem}
\end{itemize}
\end{onehalfspacing}

\datedsubsection{\textbf{催化动力学模拟包Kynetix的开发}}{2015年1月 -- 2017年1月}
\role{Python, C++, Javascript, MPI, Linux} {独自开发, 指导教师: \faLink \href{http://chem.ecust.edu.cn/2014/1211/c6655a50467/page.htm}{曹宵鸣}副教授}
\begin{onehalfspacing}
\begin{itemize}
    \item 实现催化动力学模拟程序包\emph{Kynetix}, 包括微观动力学求解器以及动力学蒙特卡洛模拟器 (未开源, GitHub private repository)
    \item 基于牛顿法求解方程组以及数值求解微分方程(ODE)的方法来对催化动力学进行微观动力学求解
    \item 基于Boostrap+Flask使用Kynetix作为后端计算程序实现了微观动力学求解web应用\faLink \href{https://github.com/PytLab/KinLab}{KinLab} (\faLink \href{http://123.206.225.154:5000/}{Demo})
    \item 完成论文\emph{Kynetix: A Software Package for chemical kinetics simulation}
\end{itemize}
\end{onehalfspacing}

\section{\faUser\ 个人开源项目}

\datedsubsection{\href{https://github.com/PytLab/gaft}{\emph{gaft}}, \emph{作者}}{2017年8月}
\begin{onehalfspacing}
\begin{itemize}
    \item \textbf{项目简介}: 通用的Python遗传算法框架, 可快速根据需求编写遗传算法求解脚本优化目标函数
    \item \textbf{主要特性}: 内置常用遗传算子和编码方式; 支持自定义遗传算子和编码; 支持自定义on-the-fly分析插件的编写; 支持自定义适应度函数; 使用MPI for Python并行化遗传算法
    \item \textbf{项目指数}: \faStar\ 129\, \faCodeFork\ 45
\end{itemize}
\end{onehalfspacing}

\datedsubsection{\href{https://github.com/PytLab/GASol}{\emph{GASol}}, \emph{作者}}{2017年11月}
\begin{onehalfspacing}
\begin{itemize}
    \item \textbf{项目简介}: 通用的C++遗传算法框架(gaft的C++版本)
    \item \textbf{主要特性}: 内置常用遗传算子和编码方式; 支持自定义遗传算子和编码; 支持自定义适应度函数; 使用MPI并行化遗传算法具有良好的强扩展效率
    \item \textbf{项目指数}: \faStar\ 3\, \faCodeFork\ 1
\end{itemize}
\end{onehalfspacing}

\datedsubsection{\href{https://github.com/PytLab/gaft}{\emph{VASPy}}, \emph{作者}}{2015年8月}
\begin{onehalfspacing}
\begin{itemize}
    \item \textbf{项目简介}: Python实现的VASP数据处理框架
    \item \textbf{主要特性}: 以VASP的输入输出文件作为基本对象来操作,大大简化了编写处理VASP数据脚本的编写
    \item \textbf{项目指数}: \faStar\ 62\, \faCodeFork\ 38
\end{itemize}
\end{onehalfspacing}

\datedsubsection{\href{https://github.com/PytLab/simpleflow}{\emph{simpleflow}}, \emph{作者}}{2018年1月}
\begin{onehalfspacing}
\begin{itemize}
    \item \textbf{项目简介}: 仿TensorFlow接口的简化版图计算框架
    \item \textbf{主要特性}: 提供了计算图中常用的操作以及求导方法; 实现了前向传播与反向传播; 实现了梯度下降优化器;
    \item \textbf{项目指数}: \faStar\ 48\, \faCodeFork\ 10
\end{itemize}
\end{onehalfspacing}

\datedsubsection{\href{https://github.com/PytLab/catplot}{\emph{catplot}}, \emph{作者}}{2015年3月}
\begin{onehalfspacing}
\begin{itemize}
    \item \textbf{项目简介}: 基于matplotlib的2D/3D抽象网格和能量曲线绘制程序
    \item \textbf{主要特性}: 使用插值算法绘制优美的能量曲线; 友好的接口方便灵活绘制2D/3D晶格示意图; 使用该程序作图发表的论文超过10篇
    \item \textbf{项目指数}: \faStar\ 19\, \faCodeFork\ 8
\end{itemize}
\end{onehalfspacing}

\section{\faHeartO\ 获奖情况}
\datedline{\textit{第一名}, xxx 比赛}{2013 年6 月}
\datedline{其他奖项}{2015}

\section{\faInfo\ 其他}
% increase linespacing [parsep=0.5ex]
\begin{itemize}[parsep=0.5ex]
  \item 技术博客: http://blog.yours.me
  \item GitHub: https://github.com/username
  \item 语言: 英语 - 熟练(TOEFL xxx)
\end{itemize}

\end{document}
