% !TEX TS-program = xelatex
% !TEX encoding = UTF-8 Unicode
% !Mode:: "TeX:UTF-8"

\documentclass{resume}
\usepackage{zh_CN-Adobefonts_external} % Simplified Chinese Support using external fonts (./fonts/zh_CN-Adobe/)
\usepackage{linespacing_fix} % disable extra space before next section
\usepackage{cite}

\begin{document}
\pagenumbering{gobble} % suppress displaying page number

\name{邵正将}

\basicInfo{
  \email{shaozhengjiang@gmail.com} \textperiodcentered\
  \phone{185-1669-0432} \textperiodcentered\
  \faGithub\ \href{https://github.com/PytLab}{PytLab} \textperiodcentered\
  \faLink\ \href{http://pytlab.org}{pytlab.org}}
 
\section{\faGraduationCap\  教育背景}
\datedsubsection{\textbf{华东理工大学\ 工业催化研究所计算化学中心}, 上海}{2014 -- 至今}
\textit{在读博士研究生}(保送直博)\ 计算化学, 预计 2019 年 6 月毕业
\datedsubsection{\textbf{华东理工大学}, 上海}{2010 -- 2014}
\textit{工学学士}\ 计算机科学与技术(辅修)
\datedsubsection{\textbf{华东理工大学}, 上海}{2010 -- 2014}
\textit{理学学士}\ 材料化学

\section{\faCogs\ 技能}
\begin{itemize}[parsep=0.5ex]
  \item 编程语言: Python, C++, C, ECMAScript, Shell
  \item 操作系统: Linux, macOS, Windows
  \item 开发工具: Vim, Git, CMake, gdb, SWIG
  \item 开发框架: MPI, TensorFlow, sklearn, Flask
  \item 数学基础: 线性代数, 遗传算法, 最优化方法
  \item 测试工具: PyUnit, CppUnit, Googletest
\end{itemize}

\section{\faUsers\ 科研项目}

\datedsubsection{\textbf{实现基于遗传算法的反应力场(ReaxFF)的通用优化方法}}{2017年8月 -- 2018年1月}
\role{Python, C++, MPI, Linux}{独自开发, 指导教师: \faLink \href{http://www.ecust.edu.cn/2016/1018/c226a56807/page.htm}{\textbf{胡培君}}教授}
\begin{onehalfspacing}
\begin{itemize}
  \item 实现基于MPI的分布式通用Python遗传算法优化框架GAFT, https://github.com/PytLab/gaft
  \item 实现基于MPI的分布式通用C++遗传算法求解器GASol, https://github.com/PytLab/GASol
  \item 基于GAFT实现反应力场ReaxFF的力场优化程序(未开源, GitHub private repository)
  \item 优化PtO催化体系的反应力场势能面同第一性原理能量均方根误差达到0.1eV以下
\end{itemize}
\end{onehalfspacing}

\datedsubsection{\textbf{\LaTeX\ 简历模板}}{2015 年5月 -- 至今}
\role{\LaTeX, Python}{个人项目}
\begin{onehalfspacing}
优雅的 \LaTeX\ 简历模板, https://github.com/billryan/resume
\begin{itemize}
  \item 容易定制和扩展
  \item 完善的 Unicode 字体支持,使用 \XeLaTeX\ 编译
  \item 支持 FontAwesome 4.5.0
\end{itemize}
\end{onehalfspacing}

% Reference Test
%\datedsubsection{\textbf{Paper Title\cite{zaharia2012resilient}}}{May. 2015}
%An xxx optimized for xxx\cite{verma2015large}
%\begin{itemize}
%  \item main contribution
%\end{itemize}

\section{\faHeartO\ 获奖情况}
\datedline{\textit{第一名}, xxx 比赛}{2013 年6 月}
\datedline{其他奖项}{2015}

\section{\faInfo\ 其他}
% increase linespacing [parsep=0.5ex]
\begin{itemize}[parsep=0.5ex]
  \item 技术博客: http://blog.yours.me
  \item GitHub: https://github.com/username
  \item 语言: 英语 - 熟练(TOEFL xxx)
\end{itemize}

%% Reference
%\newpage
%\bibliographystyle{IEEETran}
%\bibliography{mycite}
\end{document}
