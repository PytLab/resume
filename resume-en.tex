% !TEX TS-program = xelatex
% !TEX encoding = UTF-8 Unicode
% !Mode:: "TeX:UTF-8"

\documentclass{resume}
\usepackage{zh_CN-Adobefonts_external} % Simplified Chinese Support using external fonts (./fonts/zh_CN-Adobe/)
\usepackage{linespacing_fix} % disable extra space before next section
\usepackage{cite}

\begin{document}
\pagenumbering{gobble} % suppress displaying page number

\name{Shao Zheng-jiang}

\basicInfo{
  \email{shaozhengjiang@gmail.com} \textperiodcentered\
  \phone{185-1669-0432} \textperiodcentered\
  \faGithub\ \href{https://github.com/PytLab}{PytLab} \textperiodcentered\
  \faLink\ \href{http://pytlab.org}{pytlab.org}}
 
\section{\faGraduationCap\  Education}
\datedsubsection{\textbf{Center for Computational Chemistry, ECUST}, Shanghai}{2014 -- Present}
\textit{PhD Candidate} in Computational Chemistry, expected June 2019
\datedsubsection{\textbf{East China University of Science and Technology (ECUST)}, Shanghai}{2010 -- 2014}
\textit{B.E.} in Computer Science and Technology (Second Major)
\datedsubsection{\textbf{East China University of Science and Technology (ECUST)}, Shanghai}{2010 -- 2014}
\textit{B.S.} in Material Chemistry

\section{\faCogs\ Skills}
\begin{itemize}[parsep=0.5ex]
    \item \textbf{Programming Languages}: Python, C++, C, ECMAScript, Shell
    \item \textbf{Operating System}: Linux, macOS, Windows
    \item \textbf{Development Tools}: Vim, Git, CMake, gdb, SWIG, Jupyter notebook
    \item \textbf{Development Framework}: MPI, TensorFlow, sklearn, Flask
    \item \textbf{Mathmetical Basis}: Linear Algebra, Genetic Algorithm, Optimization Methods
    \item \textbf{Testing Tools}: PyUnit, CppUnit, Googletest
    \item \textbf{English Ability}: CET6 (568)
\end{itemize}

\section{\faUsers\ Scientific research projects}

\datedsubsection{\textbf{Genetic Algorithm based general optimization method for reactive force field (ReaxFF)}}{Aug 2017 -- Jan 2018}
\role{Python, C++, MPI, Linux}{Developed independently, tutor: \faLink \href{http://www.ecust.edu.cn/2016/1018/c226a56807/page.htm}{Prof.\ Peijun Hu}}
\begin{onehalfspacing}
\begin{itemize}
    \item Developed the general genetic algorithm framework for Python: \faLink \href{https://github.com/PytLab/gaft}{GAFT}\ (\faStar\ 129 \ \faCodeFork\ 45)
  \item Developed the force field optimization program based on GAFT (Not open source, GitHub private repository)
\end{itemize}
\end{onehalfspacing}

\datedsubsection{\textbf{A study of adsorbates interations on surface using machine learning algorithms}}{Sep 2017 -- Present}
\role{Python, scikit-learn, TensorFlow, jupyter notebook} {Tutor: \faLink \href{http://chem.ecust.edu.cn/2014/1211/c6655a50467/page.htm}{Dr.\ Xiaoming Cao}}
\begin{onehalfspacing}
\begin{itemize}
    \item Trained different regression models (SVR, DNN, CNN) using the catalysis surface configuration and formation energies data
    \item Optimized the hyper-parameters of SVR using GAFT developed by myself
\end{itemize}
\end{onehalfspacing}

\datedsubsection{\textbf{New methods for overcoming the low-barrier diffusion problems in kMC simulation}}{Sep 2016 -- May 2017}
\role{Python, C++, MPI, Linux} {Developed independently, tutor: \faLink \href{http://chem.ecust.edu.cn/2014/1211/c6655a50467/page.htm}{Dr.\ Xiaoming Cao}}
\begin{onehalfspacing}
\begin{itemize}
    \item Developed a Python/C++ kinetic Monte Carlo simulation library: KMCLibX (Not open source, GitHub private repository)
    \item Used SWIG to wrap C++ backend to provide Python interfaces and used MPI to parallize the simulation
    \item Used time-scale coupling/decoupling methods to overcome the low-barrier diffusion problem accelerating the kMC simulation significantly
    \item Finished the paper: \emph{ab initio kinetic Monte Carlo simulation of CO Oxidation: Methods for overcoming the low-barrier diffusion problem}
\end{itemize}
\end{onehalfspacing}

\datedsubsection{\textbf{Developement of chemical kinetics simulation software package: \emph{Kynetix}}}{Jan 2015 -- Jan 2017}
\role{Python, C++, Javascript, MPI, Linux} {Developed independently, tutor: \faLink \href{http://chem.ecust.edu.cn/2014/1211/c6655a50467/page.htm}{Dr.\ Xiaoming Cao}}
\begin{onehalfspacing}
\begin{itemize}
    \item Developed the chemical kinetics simulation sofware package (\emph{Kynetix}) including microkinetics solver and kinetic Monte Carlo simulator (Not open source, GitHub private repository)
    \item Develeped the web app for microkinetics simulatior using \emph{Kynetix} as the computation backend based on Boostrap and Flask: \faLink \href{https://github.com/PytLab/KinLab}{KinLab} (\faLink \href{http://123.206.225.154:5000/}{Demo})
    \item Finished the paper: \emph{Kynetix: A Software Package for chemical kinetics simulation}
\end{itemize}
\end{onehalfspacing}

\section{\faUser\ Personal Projects}

\datedsubsection{\href{https://github.com/PytLab/gaft}{\emph{gaft}}\ (\faStar\ 129\ \faCodeFork\ 45), \emph{Author}}{Ang 2017}
\begin{onehalfspacing}
\begin{itemize}
    \item \textbf{Introduction}: A general Genetic Algorithm framework in Python
    \item \textbf{Features}: Builtin genetic operators and encoder/decoders; Support custom operatior and encoder/decoder definition; Support custom on-the-fly analysis plugins; Support custom fitness function definition; Use MPI for Python to parallelize the computation
\end{itemize}
\end{onehalfspacing}

\datedsubsection{\href{https://github.com/PytLab/GASol}{\emph{GASol}}\ (\faStar\ 3\ \faCodeFork\ 1), \emph{Author}}{Nov 2017}
\begin{onehalfspacing}
\begin{itemize}
    \item \textbf{Introduction}: A general Genetic Algorithm Solver in C++
    \item \textbf{Features}: Builtin genetic operators and encoder/decoders; Support custom operatior and encoder/decoder definition; Support custom fitness function definition; Use MPI to parallelize the computation
\end{itemize}
\end{onehalfspacing}

\datedsubsection{\href{https://github.com/PytLab/VASPy}{\emph{VASPy}}\ (\faStar\ 62\ \faCodeFork\ 38), \emph{Author}}{Ang 2015}
\begin{onehalfspacing}
\begin{itemize}
    \item \textbf{Introduction}: A Python framework for processing VASP data files.
    \item \textbf{Features}: Use VASP data files as basic objects to make it much easier to do pre/post-processing for VASP data.
\end{itemize}
\end{onehalfspacing}

\datedsubsection{\href{https://github.com/PytLab/simpleflow}{\emph{simpleflow}}\ (\faStar\ 48\ \faCodeFork\ 10), \emph{Author}}{Jan 2018}
\begin{onehalfspacing}
\begin{itemize}
    \item \textbf{Introduction}: A simple TensorFlow-like graph computation framework 
    \item \textbf{Features}: Computational graph definition; Feed forward propagation and backward propagation; Gradient Optimizer
\end{itemize}
\end{onehalfspacing}

\datedsubsection{\href{https://github.com/PytLab/catplot}{\emph{catplot}}\ (\faStar\ 19\ \faCodeFork\ 8), \emph{Author}}{Mar 2015}
\begin{onehalfspacing}
\begin{itemize}
    \item \textbf{Introduction}: Python Library for Energy Profile and Abstract Grid(2D/3D) plotting
    \item \textbf{Features}: Use interpolatin algorithm to plot grateful energy profile; Provide friendly interfaces to help users to plot 2D/3D abstract grid.
\end{itemize}
\end{onehalfspacing}

\section{\faUser\ Community contributions}

\datedsubsection{\href{https://github.com/leetmaa/KMCLib}{\emph{KMCLib}}\ (\faStar\ 28\ \faCodeFork\ 24), \href{https://github.com/leetmaa/KMCLib/graphs/contributors}{\emph{Container}}}{July 2016 -- Present}
\begin{onehalfspacing}
\begin{itemize}
    \item \textbf{Introduction}: A Python/C++ kinetic Monte Carlo library developed by Leetmaa in KTH
    \item \textbf{Constribution}: Fixing bugs, efficiency improvement, dealing with issues.
\end{itemize}
\end{onehalfspacing}

\datedsubsection{\href{https://github.com/wzpan/hexo-theme-freemind}{\emph{hexo-theme-freemind}}\ (\faStar\ 292\ \faCodeFork\ 131), \href{https://github.com/wzpan/hexo-theme-freemind/graphs/contributors}{\emph{Constributor}}}{Jan 2016}
\begin{onehalfspacing}
\begin{itemize}
    \item \textbf{Introduction}: One of the most popular Hexo themes
    \item \textbf{Constribution}: Add post sharing support and setting.
\end{itemize}
\end{onehalfspacing}

\datedsubsection{\emph{PythonCoder WeChat Official Account and PythonCoder Weibo Account}, Editor}{Mar 2016 -- Nov 2017}

\datedsubsection{\href{http://www.jobbole.com/members/shaozhengjiang/}{\emph{jobbole}}, \href{https://juejin.im/user/59f6fb0ff265da432e5b4ad0/posts}{\emph{juejin}}, \href{https://zhuanlan.zhihu.com/zimei}{\emph{PyChina}}, Columnist}{Mar 2016 -- Present}

\section{\faTrophy\ Awards}
\datedline{\textit{Outstanding graduation thesis and Outstanding graduates}, school-level}{June 2014}
\datedline{\textit{Second-class comprehensive scholarship, Chemical Zone Student Innovation Award}, school-level}{Sep 2013}
\datedline{\textit{The 6th China Undergraduate Computer Design Competition}, \href{http://www.jsjds.org/Article_Show.asp?ArticleID=245}{Second prize}}{July 2013}

\end{document}
