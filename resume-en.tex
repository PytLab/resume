% !TEX TS-program = xelatex
% !TEX encoding = UTF-8 Unicode
% !Mode:: "TeX:UTF-8"

\documentclass{resume}
\usepackage{zh_CN-Adobefonts_external} % Simplified Chinese Support using external fonts (./fonts/zh_CN-Adobe/)
\usepackage{linespacing_fix} % disable extra space before next section
\usepackage{cite}

\begin{document}
\pagenumbering{gobble} % suppress displaying page number

\name{Shao Zheng-jiang}

\basicInfo{
  \email{shaozhengjiang@gmail.com} \textperiodcentered\
  \phone{185-1669-0432} \textperiodcentered\
  \faGithub\ \href{https://github.com/PytLab}{PytLab} \textperiodcentered\
  \faLink\ \href{http://pytlab.org}{pytlab.org}}
 
\section{\faGraduationCap\  Education}
\datedsubsection{\textbf{Center for Computational Chemistry, ECUST}, Shanghai}{2014 -- Present}
\textit{PhD Candidate} in Computational Chemistry, expected June 2019
\datedsubsection{\textbf{East China University of Science and Technology (ECUST)}, Shanghai}{2010 -- 2014}
\textit{B.E.} in Computer Science and Technology (Second Major)
\datedsubsection{\textbf{East China University of Science and Technology (ECUST)}, Shanghai}{2010 -- 2014}
\textit{B.S.} in Material Chemistry

\section{\faCogs\ Skills}
\begin{itemize}[parsep=0.5ex]
    \item \textbf{Programming Languages}: Python, C++, C, ECMAScript, Shell
    \item \textbf{Operating System}: Linux, macOS, Windows
    \item \textbf{Development Tools}: Vim, Git, CMake, gdb, SWIG
    \item \textbf{Development Framework}: MPI, TensorFlow, sklearn, Flask
    \item \textbf{Mathmetical Basis}: Linear Algebra, Genetic Algorithm, Optimization Methods
    \item \textbf{Testing Tools}: PyUnit, CppUnit, Googletest
    \item \textbf{English Ability}: CET6 (568)
\end{itemize}

\section{\faUsers\ Scientific research projects}

\datedsubsection{\textbf{Genetic Algorithm based general optimization method for reactive force field (ReaxFF)}}{Aug 2017 -- Jan 2018}
\role{Python, C++, MPI, Linux}{Developed independently, tutor: \faLink \href{http://www.ecust.edu.cn/2016/1018/c226a56807/page.htm}{Prof.\ Peijun Hu}}
\begin{onehalfspacing}
\begin{itemize}
    \item Developed the general genetic algorithm framework for Python: \faLink \href{https://github.com/PytLab/gaft}{GAFT}\ (\faStar\ 128 \ \faCodeFork\ 45)
  \item Developed the force field optimization program based on GAFT (Not open source, GitHub private repository)
\end{itemize}
\end{onehalfspacing}

\datedsubsection{\textbf{A study of adsorbates interations on surface using machine learning algorithms}}{Sep 2017 -- Present}
\role{Python, scikit-learn, TensorFlow, jupyter notebook} {Tutor: \faLink \href{http://chem.ecust.edu.cn/2014/1211/c6655a50467/page.htm}{Dr.\ Xiaoming Cao}}
\begin{onehalfspacing}
\begin{itemize}
    \item Trained different regression models (SVR, DNN, CNN) using the catalysis surface configuration and formation energies data
    \item Optimized the hyper-parameters of SVR using GAFT developed by myself
\end{itemize}
\end{onehalfspacing}

\datedsubsection{\textbf{New methods for overcoming the low-barrier diffusion problems in kMC simulation}}{Sep 2016 -- May 2017}
\role{Python, C++, MPI, Linux} {Developed independently, tutor: \faLink \href{http://chem.ecust.edu.cn/2014/1211/c6655a50467/page.htm}{Dr.\ Xiaoming Cao}}
\begin{onehalfspacing}
\begin{itemize}
    \item Developed a Python/C++ kinetic Monte Carlo simulation library: KMCLibX (Not open source, GitHub private repository)
    \item Used SWIG to wrap C++ backend to provide Python interfaces and used MPI to parallize the simulation
    \item Used time-scale coupling/decoupling methods to overcome the low-barrier diffusion problem accelerating the kMC simulation significantly
    \item Finished the paper: \emph{ab initio kinetic Monte Carlo simulation of CO Oxidation: Methods for overcoming the low-barrier diffusion problem}
\end{itemize}
\end{onehalfspacing}

\datedsubsection{\textbf{Developement of chemical kinetics simulation software package: \emph{Kynetix}}}{Jan 2015 -- Jan 2017}
\role{Python, C++, Javascript, MPI, Linux} {Developed independently, tutor: \faLink \href{http://chem.ecust.edu.cn/2014/1211/c6655a50467/page.htm}{Dr.\ Xiaoming Cao}}
\begin{onehalfspacing}
\begin{itemize}
    \item Developed the chemical kinetics simulation sofware package (\emph{Kynetix}) including microkinetics solver and kinetic Monte Carlo simulator (Not open source, GitHub private repository)
    \item Develeped the web app for microkinetics simulatior using \emph{Kynetix} as the computation backend based on Boostrap and Flask: \faLink \href{https://github.com/PytLab/KinLab}{KinLab} (\faLink \href{http://123.206.225.154:5000/}{Demo})
    \item Finished the paper: \emph{Kynetix: A Software Package for chemical kinetics simulation}
\end{itemize}
\end{onehalfspacing}

\section{\faUser\ Personal Projects}

\datedsubsection{\href{https://github.com/PytLab/gaft}{\emph{gaft}}\ (\faStar\ 129\ \faCodeFork\ 45), \emph{作者}}{2017年8月}
\begin{onehalfspacing}
\begin{itemize}
    \item \textbf{项目简介}: 通用的Python遗传算法框架, 可快速根据需求编写遗传算法求解脚本优化目标函数
    \item \textbf{主要特性}: 内置常用遗传算子和编码方式; 支持自定义遗传算子和编码; 支持自定义on-the-fly分析插件的编写; 支持自定义适应度函数; 使用MPI for Python并行化遗传算法
\end{itemize}
\end{onehalfspacing}

\datedsubsection{\href{https://github.com/PytLab/GASol}{\emph{GASol}}\ (\faStar\ 3\ \faCodeFork\ 1), \emph{作者}}{2017年11月}
\begin{onehalfspacing}
\begin{itemize}
    \item \textbf{项目简介}: 通用的C++遗传算法框架(gaft的C++版本)
    \item \textbf{主要特性}: 内置常用遗传算子和编码方式; 支持自定义遗传算子和编码; 支持自定义适应度函数; 使用MPI并行化遗传算法具有良好的强扩展效率
\end{itemize}
\end{onehalfspacing}

\datedsubsection{\href{https://github.com/PytLab/gaft}{\emph{VASPy}}\ (\faStar\ 62\ \faCodeFork\ 38), \emph{作者}}{2015年8月}
\begin{onehalfspacing}
\begin{itemize}
    \item \textbf{项目简介}: Python实现的VASP数据处理框架(VASP是一种量子力学计算商业软件)
    \item \textbf{主要特性}: 以VASP的输入输出文件作为基本对象来操作,大大简化了编写处理VASP数据脚本的编写
\end{itemize}
\end{onehalfspacing}

\datedsubsection{\href{https://github.com/PytLab/simpleflow}{\emph{simpleflow}}\ (\faStar\ 48\ \faCodeFork\ 10), \emph{作者}}{2018年1月}
\begin{onehalfspacing}
\begin{itemize}
    \item \textbf{项目简介}: 仿TensorFlow接口的简化版图计算框架
    \item \textbf{主要特性}: 提供了计算图中常用的操作以及求导方法; 实现了前向传播与反向传播; 实现了梯度下降优化器;
\end{itemize}
\end{onehalfspacing}

\datedsubsection{\href{https://github.com/PytLab/catplot}{\emph{catplot}}\ (\faStar\ 19\ \faCodeFork\ 8), \emph{作者}}{2015年3月}
\begin{onehalfspacing}
\begin{itemize}
    \item \textbf{项目简介}: 基于matplotlib的2D/3D抽象网格和能量曲线绘制程序
    \item \textbf{主要特性}: 使用插值算法绘制优美的能量曲线; 友好的接口方便灵活绘制2D/3D晶格示意图; 使用该程序作图发表的论文超过10篇
\end{itemize}
\end{onehalfspacing}

\section{\faUser\ 社区贡献}

\datedsubsection{\href{https://github.com/leetmaa/KMCLib}{\emph{KMCLib}}\ (\faStar\ 28\ \faCodeFork\ 24), \href{https://github.com/leetmaa/KMCLib/graphs/contributors}{\emph{维护者}}}{2016年7月 -- 至今}
\begin{onehalfspacing}
\begin{itemize}
    \item \textbf{项目简介}: 瑞典皇家理工学院Leetmaa开发的动力学蒙特卡洛Python/C++库
    \item \textbf{贡献}: 修复bug, 效率改进, 解决使用者提出的问题
\end{itemize}
\end{onehalfspacing}

\datedsubsection{\href{https://github.com/wzpan/hexo-theme-freemind}{\emph{hexo-theme-freemind}}\ (\faStar\ 292\ \faCodeFork\ 131), \href{https://github.com/wzpan/hexo-theme-freemind/graphs/contributors}{\emph{贡献者}}}{2016年1月}
\begin{onehalfspacing}
\begin{itemize}
    \item \textbf{项目简介}: 最受欢迎的 Hexo 博客主题之一
    \item \textbf{贡献}: 添加文章分享功能和设置
\end{itemize}
\end{onehalfspacing}

\datedsubsection{\emph{Python开发者微信公众号、Python开发者微博}, 编辑}{2016年3月 -- 2017年11月}

\datedsubsection{\href{http://www.jobbole.com/members/shaozhengjiang/}{\emph{伯乐在线}}, \href{https://juejin.im/user/59f6fb0ff265da432e5b4ad0/posts}{\emph{掘金社区}}, \href{https://zhuanlan.zhihu.com/zimei}{\emph{Python中文社区}}, 专栏作者}{2016年3月 -- 至今}

\section{\faTrophy\ 获奖情况}
\datedline{\textit{优秀毕业论文, 优秀毕业生}, 校级}{2014年6月}
\datedline{\textit{校综合奖学金二等, 化工区学生创新奖}, 校级}{2013年9月}
\datedline{\textit{第六届中国大学生计算机设计大赛}, \href{http://www.jsjds.org/Article_Show.asp?ArticleID=245}{二等奖}}{2013年7月}

\end{document}
