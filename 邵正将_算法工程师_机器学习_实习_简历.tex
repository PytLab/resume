% !TEX TS-program = xelatex
% !TEX encoding = UTF-8 Unicode
% !Mode:: "TeX:UTF-8"

\documentclass{resume}
\usepackage{zh_CN-Adobefonts_external} % Simplified Chinese Support using external fonts (./fonts/zh_CN-Adobe/)
\usepackage{linespacing_fix} % disable extra space before next section
\usepackage{cite}

\begin{document}
\pagenumbering{gobble} % suppress displaying page number

\name{邵正将}

\basicInfo{
  \email{shaozhengjiang@gmail.com} \textperiodcentered\
  \phone{185-1669-0432} \textperiodcentered\
  \faGithub\ \href{https://github.com/PytLab}{PytLab} \textperiodcentered\
  \faLink\ \href{http://pytlab.org}{pytlab.org}}
 
\section{\faGraduationCap\  教育背景}
\datedsubsection{\textbf{华东理工大学\ 工业催化研究所计算中心}, 上海}{2014 -- 至今}
\textit{在读博士研究生}(保送直博)\ 计算化学, 预计 2019 年 6 月毕业
\datedsubsection{\textbf{华东理工大学}, 上海}{2010 -- 2014}
\textit{工学学士}\ 计算机科学与技术(双学位)
\datedsubsection{\textbf{华东理工大学}, 上海}{2010 -- 2014}
\textit{理学学士}\ 材料化学

\section{\faCogs\ 技能}
\begin{itemize}[parsep=0.5ex]
  \item 编程语言: Python, C++, C, ECMAScript, Shell
  \item 操作系统: Linux, macOS, Windows
  \item 开发工具: Vim, Git, CMake, gdb, SWIG, Jupyter Notebook
  \item 开发框架: MPI, TensorFlow, scikit-learn, Flask
  \item 数学基础: 线性代数, 遗传算法, 最优化方法
  \item 测试工具: PyUnit, CppUnit, Googletest
  \item 英语能力: CET6 (568)
\end{itemize}

\section{\faUsers\ 科研项目}

\datedsubsection{\textbf{实现基于遗传算法的反应力场(ReaxFF)的通用优化方法}}{2017年8月 -- 2018年1月}
\role{Python, C++, MPI, Linux}{独自开发, 指导教师: \faLink \href{http://www.ecust.edu.cn/2016/1018/c226a56807/page.htm}{胡培君}教授}
\begin{onehalfspacing}
\begin{itemize}
    \item 实现基于MPI的分布式通用Python遗传算法优化框架\faLink \href{https://github.com/PytLab/gaft}{GAFT}\ (\faStar\ 135 \ \faCodeFork\ 47)
    \item 基于GAFT实现反应力场ReaxFF的力场优化程序(未开源, GitHub private repository)
\end{itemize}
\end{onehalfspacing}

\datedsubsection{\textbf{基于机器学习算法的催化剂表面物种相互作用的研究}}{2017年9月 -- 至今}
\role{Python, scikit-learn, TensorFlow, jupyter notebook} {指导教师: \faLink \href{http://chem.ecust.edu.cn/2014/1211/c6655a50467/page.htm}{曹宵鸣}副教授}
\begin{onehalfspacing}
\begin{itemize}
    \item 尝试使用不同机器学习模型(SVR, DNN, CNN)构建催化剂二维表面构型与形成能的回归关系
    \item 采用构型对称性扩大数据集并使用自己开发的遗传算法框架GAFT对SVR模型超参数进行优化
\end{itemize}
\end{onehalfspacing}

\datedsubsection{\textbf{\emph{ab initio} kMC模拟中克服物种迁移低能垒问题的新方法研究}}{2016年9月 -- 2017年5月}
\role{Python, C++, MPI, Linux} {独自开发, 指导教师: \faLink \href{http://chem.ecust.edu.cn/2014/1211/c6655a50467/page.htm}{曹宵鸣}副教授}
\begin{onehalfspacing}
\begin{itemize}
    \item 实现了Python/C++动力学蒙特卡洛(kMC)模拟库KMCLibX(未开源, GitHub private repository)
    \item 使用SWIG为kMC C++库添加Python接口提升库的灵活性并使用MPI进行并行处理
    \item 使用time-scale coupling/decoupling 方法克服了物种迁移低能垒问题大幅度加速模拟的时间推进
    \item 完成论文\emph{ab initio kinetic Monte Carlo simulation of CO Oxidation: Methods for overcoming the low-barrier diffusion problem}
\end{itemize}
\end{onehalfspacing}

\datedsubsection{\textbf{催化动力学模拟包Kynetix的开发}}{2015年1月 -- 2017年1月}
\role{Python, C++, Javascript, MPI, Linux} {独自开发, 指导教师: \faLink \href{http://chem.ecust.edu.cn/2014/1211/c6655a50467/page.htm}{曹宵鸣}副教授}
\begin{onehalfspacing}
\begin{itemize}
    \item 实现催化动力学模拟程序包\emph{Kynetix}, 包括微观动力学求解器以及动力学蒙特卡洛模拟器 (未开源, GitHub private repository)
    \item 基于Boostrap+Flask使用Kynetix作为后端计算程序实现了微观动力学求解web应用\faLink \href{https://github.com/PytLab/KinLab}{KinLab} (\faLink \href{http://123.206.225.154:5000/}{Demo})
    \item 完成论文\emph{Kynetix: A Software Package for chemical kinetics simulation}
\end{itemize}
\end{onehalfspacing}

\section{\faUser\ 个人开源项目}

\datedsubsection{\href{https://github.com/PytLab/gaft}{\emph{gaft}}\ (\faStar\ 135\ \faCodeFork\ 47), \emph{作者}}{2017年8月}
\begin{onehalfspacing}
\begin{itemize}
    \item \textbf{项目简介}: 通用的Python遗传算法框架, 可快速根据需求编写遗传算法求解脚本优化目标函数
    \item \textbf{主要特性}: 内置常用遗传算子和编码方式; 支持自定义遗传算子和编码; 支持自定义on-the-fly分析插件的编写; 支持自定义适应度函数; 使用MPI for Python并行化遗传算法
\end{itemize}
\end{onehalfspacing}

\datedsubsection{\href{https://github.com/PytLab/GASol}{\emph{GASol}}\ (\faStar\ 3\ \faCodeFork\ 1), \emph{作者}}{2017年11月}
\begin{onehalfspacing}
\begin{itemize}
    \item \textbf{项目简介}: 通用的C++遗传算法框架(gaft的C++版本)
    \item \textbf{主要特性}: 内置常用遗传算子和编码方式; 支持自定义遗传算子和编码; 支持自定义适应度函数; 使用MPI并行化遗传算法具有良好的强扩展效率
\end{itemize}
\end{onehalfspacing}

\datedsubsection{\href{https://github.com/PytLab/gaft}{\emph{VASPy}}\ (\faStar\ 62\ \faCodeFork\ 38), \emph{作者}}{2015年8月}
\begin{onehalfspacing}
\begin{itemize}
    \item \textbf{项目简介}: Python实现的VASP数据处理框架(VASP是一种量子力学计算商业软件)
    \item \textbf{主要特性}: 以VASP的输入输出文件作为基本对象来操作,大大简化了编写处理VASP数据脚本的编写
\end{itemize}
\end{onehalfspacing}

\datedsubsection{\href{https://github.com/PytLab/simpleflow}{\emph{simpleflow}}\ (\faStar\ 49\ \faCodeFork\ 13), \emph{作者}}{2018年1月}
\begin{onehalfspacing}
\begin{itemize}
    \item \textbf{项目简介}: 仿TensorFlow接口的简化版图计算框架
    \item \textbf{主要特性}: 提供了计算图中常用的操作以及求导方法; 实现了前向传播与反向传播; 实现了梯度下降优化器;
\end{itemize}
\end{onehalfspacing}

\datedsubsection{\href{https://github.com/PytLab/catplot}{\emph{catplot}}\ (\faStar\ 19\ \faCodeFork\ 8), \emph{作者}}{2015年3月}
\begin{onehalfspacing}
\begin{itemize}
    \item \textbf{项目简介}: 基于matplotlib的2D/3D抽象网格和能量曲线绘制程序
    \item \textbf{主要特性}: 使用插值算法绘制优美的能量曲线; 友好的接口方便灵活绘制2D/3D晶格示意图; 使用该程序作图发表的论文超过10篇
\end{itemize}
\end{onehalfspacing}

\section{\faCodeFork\ 社区贡献}

\datedsubsection{\href{https://github.com/leetmaa/KMCLib}{\emph{KMCLib}}\ (\faStar\ 28\ \faCodeFork\ 24), \href{https://github.com/leetmaa/KMCLib/graphs/contributors}{\emph{维护者}}}{2016年7月 -- 至今}
\begin{onehalfspacing}
\begin{itemize}
    \item \textbf{项目简介}: 瑞典皇家理工学院Leetmaa开发的动力学蒙特卡洛Python/C++库
    \item \textbf{贡献}: 修复bug, 效率改进, 解决使用者提出的问题
\end{itemize}
\end{onehalfspacing}

\datedsubsection{\href{https://github.com/wzpan/hexo-theme-freemind}{\emph{hexo-theme-freemind}}\ (\faStar\ 292\ \faCodeFork\ 131), \href{https://github.com/wzpan/hexo-theme-freemind/graphs/contributors}{\emph{贡献者}}}{2016年1月}
\begin{onehalfspacing}
\begin{itemize}
    \item \textbf{项目简介}: 最受欢迎的 Hexo 博客主题之一
    \item \textbf{贡献}: 添加文章分享功能和设置
\end{itemize}
\end{onehalfspacing}

\datedsubsection{\emph{Python开发者微信公众号、Python开发者微博}, 编辑}{2016年3月 -- 2017年11月}

\datedsubsection{\href{http://www.jobbole.com/members/shaozhengjiang/}{\emph{伯乐在线}}, \href{https://juejin.im/user/59f6fb0ff265da432e5b4ad0/posts}{\emph{掘金社区}}, \href{https://zhuanlan.zhihu.com/zimei}{\emph{Python中文社区}}, 专栏作者}{2016年3月 -- 至今}

\section{\faTrophy\ 获奖情况}
\datedline{\textit{优秀毕业论文, 优秀毕业生}, 校级}{2014年6月}
\datedline{\textit{校综合奖学金二等, 化工区学生创新奖}, 校级}{2013年9月}
\datedline{\textit{第六届中国大学生计算机设计大赛}, \href{http://www.jsjds.org/Article_Show.asp?ArticleID=245}{二等奖}}{2013年7月}

\end{document}
